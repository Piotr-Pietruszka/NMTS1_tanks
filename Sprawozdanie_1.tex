\documentclass{article}

%\usepackage[T1]{fontenc}
%\usepackage[utf8]{inputenc}
%\usepackage{polski}
\usepackage[utf8]{inputenc}
\usepackage{polski}
\usepackage[polish]{babel}
\usepackage{mathtools}
\usepackage[thinc]{esdiff}
\usepackage{graphicx} 

\usepackage{float}

\usepackage{geometry}

\begin{document}
\newgeometry{tmargin =3cm, bmargin=3cm, lmargin=3cm, rmargin=3cm}
\begin{tabular}{|c|c|c|}
\hline 
\multicolumn{3}{|c|}{\huge Sprawozdanie } \\ 
\hline 
\multicolumn{3}{|c|}{\LARGE Projekt 1  NMTS} \\ 
\hline 
\Large Przygotowali: &\Large Piotr Pietruszka 171842 &\Large Marcin Wankiewicz 172118  \\ 
\hline 
\Large Kierunek: ACiR  \\ 
\hline 
 
\end{tabular} 

\section{Modele stanowe bez kontrolera}
\begin{equation}\label{eq:q_y}
 \begin{array}{l}
  q = \alpha_1 \rho_f g (H_1 - H_2) \\
  y = \alpha_2 \rho_f g H_2
\end{array}
\end{equation}


\begin{equation}\label{eq:H_1_2}
 \begin{array}{l}
  \diff{H_1}{t} = \frac{u-q}{S_1} \\
  \diff{H_2}{t} = \frac{q-y}{S_2}
 \end{array}
\end{equation}

Z \ref{eq:H_1_2} oraz \ref{eq:q_y}, przyjmując $H_1$ i $H_2$ jako zmienne stanu, uzyskano model stanowy \ref{eq:ss_cont}.
\begin{equation}\label{eq:stan_disc}
 \begin{array}{l}
  \mathbf{x}[n+1] = \mathbf{A}\mathbf{x}[n] + \mathbf{B}\mathbf{u}[n] \\
  \mathbf{y}[n]= \mathbf{C} \mathbf{x}[n] + \mathbf{D}\mathbf{u}[n]
\end{array}
\end{equation}


\begin{equation}\label{eq:stan_disc}
 \begin{array}{l}
  \mathbf{x}[n+1] = \mathbf{A}\mathbf{x}[n] + \mathbf{B}\mathbf{u}[n] \\
  \mathbf{y}[n]= \mathbf{C} \mathbf{x}[n] + \mathbf{D}\mathbf{u}[n]
\end{array}
\end{equation}

\begin{equation}\label{eq:ss_cont}
 \begin{array}{l}
 
 \begin{bmatrix} \dot{H_1} \\ \dot{H_2} \end{bmatrix}  = 
 \begin{bmatrix} -\frac{\alpha_1 \rho_f g}{S_1} & \frac{\alpha_1 \rho_f g}{S_1}\\ 
  \frac{\alpha_1 \rho_f g}{S_2} & -\frac{\alpha_1 \rho_f g + \alpha_2 \rho_f g}{S_2} 
 \end{bmatrix}   \begin{bmatrix} H_1 \\ H_2 \end{bmatrix} + 
 \begin{bmatrix}\frac{1}{S_1} \\ 0\end{bmatrix}u  \\
 
 y[n]  = \begin{bmatrix} 0 & \alpha_1 \rho_f g \end{bmatrix}
 \begin{bmatrix} H_1 \\ H_2 \end{bmatrix} + 
 \begin{bmatrix} 0 \end{bmatrix} u[n]
 
 \end{array}
\end{equation}



\end{document}


